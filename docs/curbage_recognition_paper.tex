\documentclass[11pt]{article}
\usepackage[utf8]{inputenc}
\usepackage[english]{babel}
\usepackage[font=small,labelfont=bf]{caption}
\usepackage{geometry}
%\usepackage[sort&compress, numbers, super]{natbib}
\usepackage{pxfonts}
\usepackage{graphicx}
\usepackage{newfloat}
\usepackage{setspace}
\usepackage{hyperref}
\usepackage{lineno}
\usepackage{placeins}
\usepackage[
sorting=none, 
style=apa
]{biblatex}
\usepackage{multirow}
\usepackage[utf8]{inputenc}
\usepackage[T1]{fontenc}
\usepackage{array, booktabs, ragged2e}
\addbibresource{curbage_recognition_bibliography.bib}
\usepackage{amsmath}
%\usepackage[nofiglist, nomarkers, fighead]{endfloat}
%\renewcommand{\includegraphics}[2][]{}

\usepackage{fancyhdr}
\pagestyle{fancy}
\lhead{AGING IMPACTS MEMORY FOR PERCEPTUAL DETAILS}
%\rhead{\thepage}
\renewcommand{\headrulewidth}{0.0pt}
%\renewcommand{\footrulewidth}{0.4pt}

\doublespacing
\linenumbers


\title{Aging impacts memory for perceptual, but not narrative event details}

\author{Angelique I. Delarazan\textsuperscript{1, *}, Charan Ranganath\textsuperscript{2}, and Zachariah M. Reagh\textsuperscript{1}\\\\ \textsuperscript{1}Department of Psychological and Brain Sciences\\
Washington University in St. Louis \\
St. Louis, MO 63130, USA\\
\textsuperscript{2}Center for Neuroscience\\
University of California, Davis \\
Davis, CA 95616, USA\\
\textsuperscript{*}Corresponding author: a.delarazan@wustl.edu}

\date{}

\begin{document}
\begin{titlepage}
  \maketitle
  \thispagestyle{empty}
  \end{titlepage}

\begin{abstract}
Memory declines over the course of healthy aging. However, memory is not a monolith, and draws from different kinds of representations. Historically, much of our understanding of age-related memory decline arise from recognition of items that tend to focus on isolated studied items. In contrast, real-life events are remembered as narratives, and this kind of information is often missed in typical memory recognition studies.  Here, we designed a task to tax mnemonic discrimination of event details, directly contrasting perceptual and narrative memory. Older and younger adults watched an episode of a television show and later completed an Old/New recognition test featuring targets, novel foils, and similar lures in narrative and perceptual domains. While we observed no age-related differences on basic recognition of repeated targets and novel foils, older adults showed a deficit in correctly rejecting perceptual, but not narrative lures. Further analyses split the older adult group into low versus high scorers on the Montreal Cognitive Assessment (MoCA), revealing that age-related deficits in discrimination performance of perceptual lures were largely driven by older adults with low MoCA scores. These findings provide insights into the vulnerability of different memory domains in aging, and may be useful in characterizing individuals at risk for pathological cognitive decline.

\textit{Keywords: memory; aging; perceptual; narrative; discrimination} 
\end{abstract}

\section*{Introduction}
Memory decline is among the most commonly reported cognitive changes with aging (\cite{craik_memory_1994}). However, memory is not a unitary phenomenon, and memory performance can often be based on multiple processes and types of representations. Much of our understanding about memory and its effects on aging arise from paradigms that often involve learning a list of words or a set of images and later retrieving these items (\cite{fraundorf_aging_2019}). Tests of this kind largely assess recognition memory processes for individual items—snapshots of perceptual experience in the context of a laboratory experiment.

Recognition memory has been well-studied in older adults, with studies showing that aging associated with a relative loss of memory for details and a sparing of memory for gist (\cite{schacter_false_1997}). A recent paradigm designed to explicitly test the specificity or fidelity of recognition memory is the Mnemonic Similarity Task (MST), which is argued to place strong demands on pattern separation (\cite{stark_task_2013, stark_mnemonic_2019}). Pattern separation refers to the process whereby similar representations are transformed into distinct, nonoverlapping representations (\cite{yassa_pattern_2011}). Though there are many variants of the task, typical MST paradigms involve an incidental encoding task, such as making \textit{indoor} or \textit{outdoor} judgments for pictures of everyday objects, and then a surprise recognition memory test whereby participants are tasked to identify exact repetitions of previously encoded objects (targets), new objects (foils), and objects that are perceptually similar images seen during the encoding task (lures) as \textit{Old} or \textit{New}. Successful discrimination of lure items requires being able to represent and accurately identify similar items as distinct and is thought to drive pattern separation in the hippocampus. 

Importantly, the MST is well established to be sensitive to hippocampal dysfunction. The ability to successfully discriminate similar lures from studied targets have been found to steadily decline with age (\cite{stark_task_2013, stark_stability_2015, stark_mnemonic_2019}). These deficits are found not only pertaining to objects, but also to scenes (\cite{stark_age-related_2017}), spatial locations (\cite{reagh_spatial_2014, reagh_greater_2016, reagh_functional_2018}), and temporal lags (\cite{roberts_temporal_2014}). Cross-species evidence suggests an age-related shift away from memory driven by pattern separation and towards a process known as pattern completion (\cite{leal_neurocognitive_2015}). Pattern completion is the process of reinstating existing memories from partial or incomplete cues. In the context of the MST, older adults are more likely to endorse similar lures as old, indicating a failure to overcome interference and discriminate between the studied item and the similar item. This age-related decline in lure discrimination has been linked with neurobiological changes involving the hippocampal circuit and its surrounding projections.

Although object-recognition paradigms such as the MST offer important insights into basic memory processes and age-related changes, it is often limited to detecting perceptual changes among items or scenes. Critically, perceptual details are only one kind of information present in memories. It is possible that older adults show a reduction of highly detailed memory for perceptual information, but not for other information domains. In line with this, several studies have suggested that lure discrimination for objects is more impaired in aging than spatial or scene information (\cite{gusten_age_2021, reagh_greater_2016, reagh_functional_2018}). Another study found that lure discrimination difficulty was significantly correlated with visual perception ability, but not long-term memory or executive function (\cite{davidson_older_2019}). Interestingly, tasking participants to create meaningful associations between items directly increases memory performance as items are processed at a deeper level (Amer et al., 2018). Meaningfulness coupled with prior knowledge can serve as a memory boost for both younger and older adults (\cite{skinner_roles_2019}), and could perhaps mitigate apparent deficits in older adults as they retrieve memories.

In contrast to perceptual information, narrative structure is another type of information present in memories (\cite{leon_architecture_2016, radvansky_novel_2005}). Events in our daily experiences tend not to be isolated, but are rather bound by contextual associations and meaning. An important and unresolved question is the extent to which memory for narrative details is affected by aging, and how this might differ from perceptual memory deficits. Studies that use narratives to test memory typically task participants to recall the information from the story. Older adults sometimes show decreased recall performance compared to younger adults (\cite{rhodes_age-related_2019}). However, this may be due to the unconstrained nature of recall tasks. Recall tasks may be more taxing for older adults because they require effortful and self-initiated processing, which have been found to decrease among older adults (\cite{rhodes_age-related_2019}). Thus, it is unclear to what extent some prior results reflect difficulties with memory for narrative details, or with behavioral processes associated with self-generated recall. Furthermore, recall lacks the constraints that recognition tasks have, making it difficult to directly test differences between perceptual and narrative domains.

Testing of these two domains in a similar way allows us to gain insights into the processing of different types of memories. It has been previously proposed that narrative and perceptual information may be preferentially encoded differently in distinct cortical pathways to the hippocampus. Within the medial temporal lobes, the parahippocampal cortex and perirhinal cortex converge on the hippocampus, providing context information and item information, respectively (\cite{ranganath_binding_2010}). These processes furthermore extend into broader cortical systems. A posterior medial (PM) system has been found to encode contextual elements, such as the representation of spatial, temporal, and causal relationships. This contrasts with an anterior temporal (AT) system that encodes more information about local features, particularly pertaining to items, objects, and individual people (\cite{ranganath_two_2012, ritchey_cortico-hippocampal_2015, reagh_what_2018}). Respectively, the PM system may be relatively more involved in encoding and retrieving narrative information, and the AT system may be relatively more involved in supporting perceptual details. Under this framework, the hippocampus serves as an integration site between the two systems of different domains. Recent work suggests that the PM system may be heavily involved in processes of encoding and recalling narratives (\cite{chen_shared_2017, yeshurun_same_2017, lee_what_2020}). Furthermore, mounting evidence suggests that the PM and AT systems may be differentially vulnerable to age-related neuropathology, with AT regions being possibly more vulnerable to aging in general (\cite{maass_alzheimers_2019, reagh_functional_2018}). Thus, contrasting memory for perceptual and narrative details may offer unique perspectives into the aging brain.

Here, we designed a task to simultaneously tax mnemonic discrimination in perceptual and narrative domains. This task is analogous to traditional MST paradigms composed of an incidental encoding task followed by a recognition test. However, with the goal of tapping into mechanisms involved in encoding of the meaningful, continuous, and dynamic world that we live in, the incidental encoding task consists of watching a television sitcom (HBO’s Curb Your Enthusiasm, S01E07: AAMCO). Television shows offer a unique methodology that balances realistic scenarios while directing our attention to specific perceptual and narrative details. After encoding, participants completed an Old/New recognition test, featuring targets, foils, and similar lures in the perceptual domain, as well as a novel variant testing narrative details (\textit{see Materials, Design, and Procedure}). This allowed us to test detailed memory for perceptual and narrative information using an ecologically valid, yet constrained approach. That is, encoding involves an immersive stimulus that hinges on meaningful, and non-arbitrary, narrative organization. Additionally, although retrieval is akin to a standard recognition test, it assesses memory along two dimensions that may provide insight in how we process different memory representations for lifelike events. Performance was compared across younger and older adults for both information domains.

We predicted no differences in basic recognition of repeated targets and novel foils across age group and test domain based on prior MST results. In line with prior work suggesting greater vulnerability of AT than PM system-mediated processes in aging, we further predicted greater age-related deficits in perceptual lure discrimination than narrative lure discrimination, reflecting relatively intact memory for narrative details.

\section*{Methods}
\label{sec:methods}

\subsection*{Participants}
Forty-two participants were recruited from the Davis, California community: 21 younger adults ($M = 20.04, SD = 1.81; range = 18-25; 20 \ female$) and 21 older adults ($M = 73, SD = 7.43; range = 61-93; 10 \ female$). The study was approved by the Institutional Review Board of University of California, Davis, and all participants provided written consent before participating in the study. Younger adults were recruited from a pool of undergraduate students enrolled in psychology courses at the University of California, Davis. Inclusion criteria for younger adults included: normal hearing, normal or corrected-to-normal vision, no history of major neurological or psychiatric illness, and English as a native language. Older adults were recruited from the Davis community through online advertisement, flyers, and word of mouth. Older participants were initially contacted by phone or email for a prescreening interview. Inclusion criteria for older adults were the same as for younger adults, except the requirement of English as a native language was relaxed to include individuals who began fluency in English before age 5. All participants were naive to the stimulus, with the exception of one younger participant. Results remain the same even after the exclusion of the non-naive younger participant (\textit{see Results}). No older adults recruited for the study had formal diagnoses of cognitive or neurological disorders, including dementia or mild cognitive impairment. However, a portion of our older adult sample exhibited scores on neuropsychological tests below standardized cutoffs, which we leveraged for exploratory analyses (\textit{see Results}).

\begin{figure}
\centering
\includegraphics[width=1\textwidth]{figures/figure1.jpg}
\caption{\small \textbf{Encoding and Retrieval.} \textbf{a} Participants viewed a 26-minute episode of a television sitcom. \textbf{b} Old/New recognition task based on Narrative or Perceptual details, with order of test domain counterbalanced across participants. Each recognition task consisted of 30 targets (described or depicted moments from the video encoded), similar lures (moments described or depicted as being similar to the video encoded), and novel foils (described or depicted moments not from the video encoded).}
\label{fig:schematic}
\end{figure}

\subsection*{Materials, Design, and Procedure}
Older participants completed the following neuropsychological tests to assess for cognitive impairments: Craft21 Recall Immediate, Craft21 Recall Delayed, Montreal Cognitive Assessment (MoCA), and Multilingual Naming Test (MINT). Briefly, Craft21 assesses recall for narratives, MoCA coarsely assesses cognitive ability, and MINT assesses the ability to name objects in English. Older and younger participants viewed a twenty-six-minute episode of a television show (\textit{HBO’s Curb Your Enthusiasm, S01E07 AAMCO}) and then completed a free recall task (not included here), a recognition task, and an event segmentation task (not included here). The present analyses focus on recognition memory task performance (\textit{see Figure 1}).

Participants completed two recognition tasks based on narrative or perceptual details, wherein the narrative recognition task consisted of identifying sentences as \textit{Old} or \textit{New} via button press and the perceptual recognition task consisted of identifying images as \textit{Old} or \textit{New} via button press. We aimed to test recognition memory for highly specific information by adapting a Mnemonic Similarity Task approach. In addition to Old/New recognition, this recognition task variant includes similar lure trials which induce mnemonic interference. Critically, sentences and images were either studied targets (described or depicted moments from the video encoded), similar lures (moments described or depicted as being similar to, but differing subtly from the video encoded), and novel foils (described or depicted moments clearly not from the video encoded). Each recognition task consisted of 30 targets, 30 lures, and 30 foils. The order of narrative and perceptual recognition tasks was counterbalanced and pseudorandomized such that odd-numbered participants completed the narrative recognition task first followed by the perceptual recognition task, and even-numbered participants completed the perceptual recognition task first followed by the narrative recognition task.       

\subsection*{Analyses}
Mean proportion of correct responses for each trial type were calculated (\textit{see Table 1 and Table 3}). Recognition performance was scored as the proportion of targets, lures, and foils endorsed as being \textit{New} or \textit{Old}. Targets were scored as hits if endorsed as \textit{Old} and misses if endorsed as \textit{New}. Lures and foils were scored as correct rejections if endorsed as \textit{New} and false alarms if endorsed as \textit{Old}. Additionally, target recognition was assessed in terms of $d'$ values ($z[Target Hit Rate]$ – $z[Foil False Alarm Rate]$) derived from signal detection analysis. Older and younger adults’ recognition performance were compared using pairwise independent samples t-test within each trial type. Additionally we calculated a lure discrimination index ($LDI$; \cite{stark_task_2013, stark_mnemonic_2019}) for each subject ($p[New|Lure]$ - $p[New|Target]$). Data were analyzed using repeated-measures ANOVAs, and post-hoc contrasts were corrected for multiple comparisons using the Bonferroni method. Statistical analysis was performed in R (Version 4.0.3; \href{https://www.r-project.org/}{https://www.r-project.org/}), using the afex (\href{https://github.com/singmann/afex}{https://github.com/singmann/afex}) package for analyses of variance (ANOVAs). 

\subsubsection*{Data and Code Availability}
The full stimuli for the materials used in the present experiment, anonymized data files, coded data, R Markdown files containing the analysis scripts are available on GitHub:

\noindent{\href{https://github.com/aidelarazan/curbage\_recognition}{https://github.com/aidelarazan/curbage\_recognition}}

\subsection*{Results}
\subsubsection*{Target recognition does not significantly differ across age}
Target recognition was assessed in terms of $d’$ values derived from signal detection analysis (\textit{see Methods}). To do so, we performed a 2 x 2 ANOVA, incorporating a within-subjects factor of test domain (Narrative vs. Perceptual) and between-subjects factor of age (Older vs. Younger). The comparison revealed a significant main effect of test domain ($F(1, 40) = 26.70, p < .001$), but not for age ($F(1, 40) = 2.50, p = .12$; \textit{see Figure 2a}). Additionally, no significant interaction was observed ($F(1, 40) = 0.46, p = .50$). Pairwise comparisons revealed that participants performed significantly better at target recognition for Perceptual compared to Narrative test domains in both older ($t(40) = 3.17, p=.02\ corrected$) and younger ($t(40) = 4.14, p=.001 \ corrected$) age groups. Findings remain the same with the non-naive younger participant excluded (significant main effect of test domain ($F(1, 39) = 27.04, p < .001$), no significant main effect of age ($F(1, 39) = 2.58, p = .12$), and no significant test domain X age interaction ($F(1, 39) = 0.63, p = .43$)). In sum, recognition of previously studied information was easier for Perceptual compared to Narrative details, but this was consistent across age groups. Importantly, basic recognition did not differ as a function of age.

\begin{table}[]
\begin{tabular}{cccccccc}
 & \multicolumn{3}{c}{\textbf{Narrative Test Domain}} & \textbf{} & \multicolumn{3}{c}{\textbf{Perceptual Test Domain}} \\ \cline{2-4} \cline{6-8} 
 & \textbf{Target}  & \textbf{Lure}  & \textbf{Foil}  & \textbf{} & \textbf{Target}   & \textbf{Lure}  & \textbf{Foil}  \\ \cline{2-4} \cline{6-8} 
Younger_{(N = 21)} & 0.86 (0.04) & 0.80 (0.08) & 0.99 (0.03) &  & 0.94 (0.03) & 0.75 (0.10) & 0.99 (0.01) \\
Older_{(N = 21)}  & 0.86 (0.07) & 0.77 (0.12) & 0.95 (0.13) &  & 0.93 (0.05) & 0.60 (0.21) & 0.97 (0.07)
\end{tabular}
\caption{Raw response proportions across age groups and trial types for both Narrative and Perceptual recognition tests. Data are presented as $mean (SD)$ proportion correct for individual trial types.}
\label{table1}
\end{table}

\begin{figure}
\centering
\includegraphics[width=1\textwidth]{figures/figure2.jpg}
\caption{\small \textbf{Average performance on target recognition ($d'$) and lure discrimination ($LDI$) across age group and test domain.} \textbf{a} Target recognition did not differ across age or test domain. \textbf{b} Significant age-related differences in Perceptual but not Narrative lure discrimination. Key: Points represent individual participants' mean performance. Bars represent average performance (+/- standard error of the mean). Significant tests: $+ \ p < .10, \ * \ p < .05, \ ** \ p < .01$.}
\label{fig:schematic}
\end{figure}

\subsubsection*{Age-related discrimination deficits for perceptual but not narrative details}
To assess discrimination of similar lure items, we performed a 2 x 2 ANOVA, with a within-subjects factor of test domain (Narrative vs. Perceptual) and between-subjects factor of age (Older vs. Younger). We corrected for response bias by calculating the LDI for each participant (\textit{see Methods}). The comparison revealed a significant main effect of age ($F(1, 40) = 6.70, p = .01$), showing that, on average, older adults were poorer at rejecting similar lures than younger adults. Additionally, results show a significant interaction between age and test domain ($F(1, 40) = 7.39, p = .01$), indicating that age group differences in lure rejection rates varied as a function of whether Narrative versus Perceptual lures were being tested (\textit{see Figure 2b}). Pairwise comparisons revealed that the interaction was driven by better lure discrimination in younger than older adults for the Perceptual ($t(40) = 3.232, p = .01 \ corrected$), but not for the Narrative lures ($t(40) = 0.82, p = 1.00 \ corrected$). No other pairwise contrasts were significant. Findings remain the same with the non-naive younger participant excluded (significant main effect of age ($F(1, 39) = 6.25, p = .02$) and significant test domain X age interaction $F(1, 39) = 6.89, p = .01$). Thus, unlike basic recognition, rejection of similar lures differed across age groups. Furthermore, this difference was driven by poorer discrimination of Perceptual, but not Narrative lures.

\subsubsection*{Narrative recognition correlates with neuropsychological tests of story recall}
A series of neuropsychological tests was collected to assess for cognitive impairments among older adults (\textit{see Table 2}). To determine whether scores were related to recognition performance, we conducted a series of correlations (Pearson's) between participant’s individual neuropsychological scores and participants’ mean $d’$ measure on Narrative and Perceptual test domains. We observed a significant positive association between $d’$ for the Craft 21 Delayed neuropsychological test and Narrative test domain ($r_{(19)} = 0.46, p = .04$) as well as Perceptual test domain ($r_{(19)} = 0.46, p = .04$). Additionally, we conducted correlations between neuropsychological test and participants’ $LDI$ score. Results showed a significant positive correlation between $LDI$ for Narrative test domain and the Craft 21 Story Immediate ($r_{(19)} = 0.50, p = .02$) and Delayed ($r_{(19)} = 0.55, p = .01$). No other correlations were statistically significant. These correlations with the Craft 21 recall test suggest that our Narrative recognition task is indeed tapping into mechanisms supporting memory for stories.

\begin{table}[]
\begin{tabular}{lccccccccccccc}
\multicolumn{1}{c}{\multirow{3}{*}{\textbf{\begin{tabular}[c]{@{}c@{}}Neuropsychological\\ Test\end{tabular}}}} &
  \multirow{3}{*}{\textbf{\begin{tabular}[c]{@{}c@{}}Score \\ $M(SD)$\end{tabular}}} &
  \textbf{} &
  \multicolumn{5}{c}{\textbf{Narrative Test Domain}} &
  \textbf{} &
  \multicolumn{5}{c}{\textbf{Perceptual Test Domain}} \\ \cline{4-8} \cline{10-14} 
\multicolumn{1}{c}{}  &              &  & \multicolumn{2}{c}{$d'$} &  & \multicolumn{2}{c}{$LDI$} &  & \multicolumn{2}{c}{$d'$} &  & \multicolumn{2}{c}{$LDI$} \\ \cline{4-5} \cline{7-8} \cline{10-11} \cline{13-14} 
\multicolumn{1}{c}{}  &              &  & r          & p         &  & r          & p          &  & r          & p         &  & r           & p         \\ \cline{1-2} \cline{4-5} \cline{7-8} \cline{10-11} \cline{13-14} 
MoCA                  & 26.07 (3.62) &  & -0.17       & 0.47      &  & 0.01      & 0.96       &  & 0.03       & 0.91      &  & 0.37        & 0.09      \\
Craft21 Immediate     & 20.48 (6.65) &  & 0.34       & 0.13     &  & 0.50       & 0.02*       &  & 0.38       & 0.09      &  & 0.11        & 0.63      \\
Craft21 Delay         & 18.52 (5.50) &  & 0.46       & 0.04*     &  & 0.55       & 0.01*      &  & 0.46       & 0.04*      &  & 0.29        & 0.20     \\
MINT    & 30.10 (1.74) &  & 0.21       & 0.36      &  & 0.12       & 0.61       &  & 0.13       & 0.58      &  & -0.17       & 0.45     
\end{tabular}
\caption{Neuropsychological test scores for older adults, and correlations between these scores and recognition performance on both tasks. Scores are presented as mean ($SD$) for neuropsychological tests. Correlations ($r$) and significance ($p$) are displayed between average score and target recognition ($d'$) and lure discrimination ($LDI$) for each task. Significant tests: $* p < 0.05$.}
\label{table2}
\end{table}

\subsubsection*{Perceptual discrimination deficits as a function of MoCA scores}
Neuropsychological tests aim to assess cognitive impairments and are often administered to older adults. Of the neuropsychological tests collected, the MoCA provides the most comprehensive test that coarsely assess for multiple cognitive processes including: short-term memory, visuospatial abilities, executive functions, attention, working memory, language and orientation to time and place. The MoCA is well-validated in testing older adults for mild cognitive impairment. MoCA scores range from 0-30, and a score of 26 or higher is generally considered “normal” (\cite{nasreddine_montreal_2005}). Our older adult sample exhibited heterogeneity in falling above or below a MoCA score of 26. In line with prior studies (\cite{pishdadian_not_2020}), we next incorporated a simple contrast of cognitive ability in our analyses via split based on MoCA score. Older participants were categorized in two groups: Low MoCA Scorers ($MoCA \ score < 26, N = 8$) and High MoCA Scorers ($MoCA \ score >= 26, N =13$), and similar comparisons between age group (Older vs. Younger) and test domain (Narrative vs. Perceptual) on $d’$ and $LDI$ measures were conducted. Thus, although we did not plan to split older adults by neuropsychological tests a priori, we leveraged this variability in our older adult sample to conduct additional analyses. Analyses reported above were repeated incorporating this subgroup split in the Older adult sample.

\begin{table}[]
\begin{tabular}{cccccccc}
                                & \multicolumn{3}{c}{\textbf{Narrative Test Domain}} & \textbf{} & \multicolumn{3}{c}{\textbf{Perceptual Test Domain}} \\ \cline{2-4} \cline{6-8} 
 & \textbf{Target} & \textbf{Lure} & \textbf{Foil} & \textbf{} & \textbf{Target} & \textbf{Lure} & \textbf{Foil} \\ \cline{2-4} \cline{6-8} 
Older_{(Low \ MoCA \ Scorer, \ N = 8)}    & 0.88 (0.08)     & 0.72 (0.11)     & 0.96 (0.12)    &           & 0.93 (0.05)     & 0.53 (0.18)     & 0.93 (0.10)     \\
Older_{(High \ MoCA \ Scorer, \ N = 13)} & 0.84 (0.07)     & 0.81 (0.12)     & 0.95 (0.14)    &           & 0.93 (0.04)     & 0.65 (0.22)     & 0.99 (0.03)    
\end{tabular}
\caption{Raw response proportions for older adults with High and Low MoCA scores. Data are presented as $mean (SD)$ proportion correct responses.}
\label{table3}
\end{table}

\begin{figure}
\centering
\includegraphics[width=.95\textwidth, scale = .5]{figures/figure3.jpg}
\caption{\small \textbf{Average performance on target recognition ($d'$) and lure discrimination ($LDI$) across age group and test domain after split on neuropsychological test performance.} Older adult sample divided into Low and High MoCA scorers, falling below or above "normal" score of 26. No age-related significant age-related differences in target recognition in both \textbf{(a)} Low MoCA Score and \textbf{(b)} High MoCA Score older adults. Significant differences in Perceptual lure discrimination were observed between young and older adults with \textbf{(c)} Low but not \textbf{(d)} High MoCA scores. No significant differences were observed for Narrative lure discrimination. Key: Points represent individual participants' mean performance. Bars represent average performance (+/- standard error of the mean). Significant tests: $* \ p < .05, \ ** \ p < .01, \ *** \ p < .001$.}
\label{fig:schematic}
\end{figure}

\subsubsection*{\textit{Perceptual recognition deficits in older adults with low MoCA scores}}
We performed a 2 x 2 ANOVA with test domain (Narrative vs. Perceptual) and age (Older vs. Younger) based on MoCA scores to determine whether there are differences in target recognition for older participants with low or high MoCA scores. For older participants with low MoCA scores, comparisons to younger participants showed a significant main effect of test domain ($F(1, 27) = 7.64, p = .001$), and a significant age X test domain interaction ($F(1, 27) = 4.96, p = .04$; \textit{see Figure 3a}). Pairwise comparisons revealed that these effects were qualified by better performance on Perceptual compared to Narrative items for younger ($t(27) = 4.75, p < .001 \ corrected$), but not older participants ($t(27) = 0.32, p = 1.00 \ corrected$). For older participants with high MoCA scores, comparisons to younger participants also showed a significant main effect of test domain ($F(1, 32) = 47.15, p < .001$; \textit{see Figure 3b}); however, corrected post-hoc pairwise comparisons revealed that high-MoCA older ($t(32) = 4.72, p < .001$) participants, like the younger participants ($t(32) = 5.11, p < .001$), were better at target recognition for Narrative but not Perceptual test domains. This difference was not observed in older participants with low MoCA scores. 

\subsubsection*{\textit{Perceptual discrimination deficits are driven by older adults with low MoCA scores}}
To determine whether there are differences in discrimination of similar lure items among older participants who scored high and low on the MoCA, we conducted similar analyses as above with $LDI$. Results show a significant main effect of age ($F(1, 27) = 12.45, p = .002$), a trending main effect of domain type ($F(1, 27) = 3.56, p = .07$), and a significant age X test domain interaction ($F(1, 27) = 11.27, p = .002$) when younger adults are compared to older participants who scored low on the MoCA (\textit{see Figure 3c}). Post-hoc pairwise comparisons revealed that the interaction was driven by poorer discrimination of similar lure items of Perceptual test domain among older adults with low MoCA scores ($t(27) = 3.08, p = 03 \ corrected$). No significant main effects of age ($F(1, 32) = 2.49, p = .12$) and domain type ($F(1, 27) = 0.10, p = .78$) were found (\textit{see Figure 3d}) were found when younger participants were compared to older participants with high MoCA scores. A trending age X domain type interaction between younger participants and older participants with high MoCA scores ($F(1, 32) = 3.23, p = .08$); however, no post-hoc pairwise comparisons were significant. In sum, these results suggest that the disproportionate deficits in perceptual lure discrimination exhibited by older adults could have been driven by the subset of older adults showing evidence of poorer global cognitive functioning.

\subsection*{Discussion}
In the present study, we sought to examine age-related changes in recognition memory for narrative and perceptual information. Younger and older participants viewed an episode of a television sitcom and later completed an Old/New recognition test consisting of targets, foils, and similar lures items that tapped into perceptual and narrative domains. Analyses revealed better performance on basic recognition of repeated targets and novel foils for perceptual compared to narrative trials across age groups. Discrimination of similar lures, however, differed across age groups, with older adults showing a deficit in correctly rejecting perceptual, but not narrative lures. Exploratory analyses divided older adults based on low versus high performance on the MoCA, allowing us to incorporate a contrast of global cognitive ability. These analyses indicated no significant differences in recognition performance between younger adults and older adults with high cognitive abilities, whereas better recognition performance on perceptual compared to narrative domain was not observed in older adults with low cognitive abilities. Furthermore, age-related deficits in lure discrimination performance among perceptual domains was mostly driven by older adults showing evidence for poorer overall cognitive ability.

Our results demonstrate the utility of including measures for more than one type of memory for the same complex stimulus. We adapted a widely used paradigm that typically aims to tax pattern separation processes in the hippocampus; however, rather than testing solely on perceptual details, as previous paradigms have done, we tested detailed recognition of narrative information as well. Memory for narrative details is often tested with spoken or written free recall, which is a different and potentially more taxing form of memory retrieval than cued recognition (\cite{craik_age_nodate}). Some findings showing age-related deficits in recall may be results affected by the difficulty of the task itself. Additionally, recall tests tend to focus largely on narrative details, and lessen the focus on perceptual details. The use of a recognition test in our design allowed us to directly assess differences between perceptual and narrative domains, while minimizing age-related differences based on the nature of the task. Thus, our results are driven by differences in the information domain (i.e., perceptual and narrative), rather than the type of test (e.g., recall versus recognition), which suggest that perceptual and narrative domains may tax distinct cognitive processes. Furthermore, these cognitive processes may rely on differentially vulnerable neural mechanisms in the aging brain, though future studies must be conducted to confirm this possibility. 	 

The hippocampus has an established role in creating and maintaining unique representations. However, these memory representations further extend into larger cortico-hippocampal networks. According to one well-supported view, memory is supported by a PM system that supports narrative information including spatiotemporal, contextual, and situational details, and an AT system that tracks more perceptually-driven information, particularly with regard to items, objects, and people (\cite{ranganath_two_2012, ritchey_cortico-hippocampal_2015}). Given that narrative structure, mediated by the PM network, provides a way to deeply encode information by allowing us to bridge overarching themes and create meaningful associations, we anticipated that older participants would perform better at recognizing narrative details aided by these associative anchors. However, differences in basic recognition performance based on test domain were driven by better (not worse) performance on perceptual information. Importantly, this effect was present across age groups, suggesting that there may be other reasons such as visual salience or difficulty level across domains that underlie this result. One possibility is that, although we explicitly tested fine-grained details across both domains, highly detailed memories may be inherently more likely to tap into perceptual representations (\cite{robin_details_2017}).

It has been argued that aging is associated with a loss of detailed memory, but a relative preservation of gist (\cite{schacter_false_1997}). This is often operationalized as retention of central, general features of studied material, but loss of specific (sometimes peripheral) information, resulting in either forgetting or false recognition due to interference (\cite{koutstaal_gist-based_1997, norman_false_1997, tun_response_nodate}. Broadly in line with this work, and in line with prior studies using MST variants (\cite{stark_task_2013, stark_stability_2015, stark_mnemonic_2019}), we find an increase in false alarms to lures but a relative preservation of target recognition in older adults. This can be viewed as a shift away from detailed memory in aging. Many prior studies informing a gist vs. detail tradeoff have used static images (\cite{stark_stability_2015}) or word lists (\cite{norman_false_1997}) as stimuli, using false alarms as the key measure. However, continuous events captured by narratives may allow us to tap into distinct mechanisms that go beyond simple visual versus verbal representations. A study by Adams and colleagues tested verbal narrative recall of younger and older adults, and showed age-related deficits in verbatim details, but that older adults showed a greater tendency toward processing a story’s interpretive meaning (\cite{adams_adult_1997}). Our results may expand on this phenomenon. Specifically, by testing both simple target recognition and lure discrimination (more taxing of detailed memory representations) across perceptual and narrative domains, our findings suggest that older participants may be more able to retain detailed memory for information that relates to a story’s meaning.

Critically, age-related discrimination deficits were limited to perceptual lures, despite perceptual target recognition being easier across both groups than narrative recognition. Increasing evidence suggests that PM and AT systems are differentially vulnerable to age-related pathology. Accumulation of tau is associated with impairment of memory processes and is predictive of Alzheimer’s disease. Early stages of Alzheimer’s disease are thought to originate in AT regions, as tau depositions accumulates in these areas (\cite{braak_frequency_nodate}). Increased tau depositions coupled with amyloid plaques later spreads in the PM regions, resulting in the progression of Alzheimer’s disease (\cite{jagust_imaging_2018, leal_subthreshold_2018}). Prior studies suggest that localization of tau and amyloid pathology in AT and PM regions are reflected in mnemonic discrimination of objects and scenes, with decreased performance on trials that require high precision in memory or perception of objects among unimpaired older adults with AT tau pathology (\cite{maass_alzheimers_2019}). Our results are in line with other findings suggesting that AT-mediated processes may be more generally vulnerable in aging (\cite{reagh_greater_2016, reagh_functional_2018}). Furthermore, evidence suggests that patients with advanced Alzheimer’s disease begin to show reliable deficits in scene lure discrimination (\cite{maass_alzheimers_2019}) and navigational ability (\cite{allison_spatial_2016}), according with a breakdown of PM-mediated contextual processes with advancing pathology. Individuals in advanced stages of aging, and particularly those with Alzheimer’s disease have also been reported to show aberrant behavior associated with event segmentation (\cite{zacks_event_2006}), a process underlying one’s ability to parse continuous information into meaningful events. Advancing age is additionally associated with changes in neural activity associated with event segmentation (\cite{reagh_aging_2020}). Together, findings of this sort suggest an increasing vulnerability of PM-mediated processes in aging, perhaps especially in Alzheimer’s disease. Although our sample does not include formally diagnosed dementia patients, our study may provide insight into future studies related to Alzheimer’s disease. Exploratory analyses that incorporated a contrast of cognitive ability indicate that declines in perceptual lure discrimination was largely driven by older adults with low MoCA scores. We speculate that this deficit is likely mediated by processes in the AT network, which may show earlier vulnerability in aging (\cite{maass_alzheimers_2019, reagh_functional_2018}). However, Alzheimer’s disease may further advance these perceptual discrimination deficits, and may also begin to feature deficits in narrative discrimination as well. Future studies can directly assess older adults on varying stages of Alzheimer’s disease using a design such as ours, to determine whether memory for narrative details is disrupted with disease progression. 

In sum, our study used a mnemonic similarity task applied to a naturalistic stimulus to show age-related deficits in perceptual, but not narrative lure discrimination. In line with several existing studies, we found domain-selective recognition deficits as a function of aging (\cite{gusten_age_2021, reagh_greater_2016, reagh_functional_2018}). These data indicate that domain-selectivity of age-related memory deficits extends to memory for continuous, lifelike information beyond simple laboratory experiments. Perceptual details, which are not anchored by narrative associations, may be particularly vulnerable in the context of aging. Additionally, our findings suggest that cognitive decline may amplify lure discrimination deficits. Testing memory for different aspects of experiences may offer important insights into memory ability in healthy and pathological aging, and a naturalistic approach offers us insights into how these processes operate in real-world situations.

\printbibliography

\section*{Acknowledgements}
We thank Alexander Garber, June Dy, Elena Markantonakis, and Ryan Bugsch for helping with data collection. We thank Erwin M. Macalalad, Brendan I. Cohn-Sheehy, and members of the Complex Memory Lab and Dynamic Memory Lab for helpful discussions and support. This research is based upon work supported by the National Science Foundation Graduate Research Fellowship Program awarded to A.I.D. under Grant No. DGE-2139839 and DGE-1745038. Additional funding from the University of California, Davis Alzheimer's Disease Research Center Pilot Grant awarded to Z.M.R. and C.R.; National Institute on Aging Grant No. T32AG050061 awarded to Z.M.R; and National Institute on Aging Grant No. 1R03AG063224-01 awarded to C.R.

\section*{Author contributions}
Conceptualization: C.R. and Z.M.R.; Methodology: A.I.D., C.R. and Z.M.R.; Investigation: A.I.D. and Z.M.R; Analysis: A.I.D. and Z.M.R.; Visualization: A.I.D.; Writing, Reviewing, and Editing: A.I.D., C.R. and Z.M.R.; Supervision: Z.M.R.

\section*{Competing interests}
The authors declare no competing interests.

\end{document}
